\documentclass[10pt]{article}
\usepackage{amsmath}
\usepackage{natbib}
\usepackage{graphicx}
\usepackage{fullpage}
\usepackage{setspace}
\usepackage{url}
\bibpunct{(}{)}{;}{a}{}{,}
\newcommand{\MC}{\multicolumn}

\begin{document}
\pagestyle{empty}
\begin{center}
\section*{Ecological Inference and Higher-Dimension Data
Management}
\emph{Olivia Lau}\footnote{Ph.D. Candidate, Department of Government,
and M.A. student, Department of Statistics, Harvard University, {\tt
olau@fas.harvard.edu}.  An alpha version of this software may be found
at \url{http://www.fas.harvard.edu/~olau/software/}.}
\mbox{$\quad\quad$} \emph{Ryan T. Moore}\footnote{Ph.D. Candidate,
Department of Government, and M.A. student, Department of Statistics,
Harvard University, {\tt rtmoore@fas.harvard.edu}} \mbox{$\quad\quad$} \emph{Mike
Kellermann}\footnote{Ph.D. Candidate, Department of Government,
Harvard University, {\tt kellerm@fas.harvard.edu}}
\end{center}

\subsection*{Abstract}

Ecological inference ({\sc ei}) takes contingency tables as the unit
of analysis.  These tables are described by marginal row and column totals (or
proportions); the goal of inference is to determine the joint
intra-table relationship between the rows and columns.  Using a common
example from political science, let the unit of analysis be a voting
precinct in a state-wide election:  
\begin{center}
\begin{tabular}{l|ccc|r}
      & Democrat & Republican & No Vote & Total \\
\hline
Black & ?        &  ?         &  ?      & 423 \\
White & ?        & ?          &  ?      & 219 \\
Hispanic & ?     & ?          &  ?      & 14 \\
\hline
Total  & 317     & 156        & 183     & 656 
\end{tabular}
\end{center}
For a given election, we estimate each cell (e.g., the number of
Blacks who voted for the Democratic candidate) for each precinct $i = 1,
\dots, I$, then aggregate across precincts to obtain election-wide results. 

While some existing R packages ({\tt MCMCpack} by Andrew Martin and
Kevin Quinn\nocite{MCMCpack} and {\tt eco} by Kosuke Imai and Ying
Lu\nocite{eco}, for example) offer functions that analyze $2 \times 2$
models, we implement more general methods that can take more than two
rows or columns.  Our package will include:
\begin{itemize}
\item Extreme case analysis, or the method of bounds, suggested by
\cite{DunDav53}. 
\item Ecological regression described in \cite{Goodman53} using both
frequentist point estimates and a Bayesian estimator that produces
correct standard errors.
\item R $\times$ C model described in \cite{RosJiaKin01} using three
estimators: a Bayesian Markov-chain Monte Carlo algorithm, maximum
likelihood, and penalized least squares.
\end{itemize}  

Since the unit of analysis (each ecological table) is a matrix rather
than a vector, studying the statistical problem of ecological
inference requires computational innovation in higher-dimension data
management.  For example, in the case of the Bayesian R $\times$ C
estimator, we need to keep track of an array of dimension:  
\begin{center}
rows $\times$ columns $\times$ precincts $\times$ simulations
\end{center}
In typical electoral data, there may be four rows (Black, White,
Hispanic, Other), three columns (Democrat, Republican, No Vote), and
11,366 precincts (in the case of Ohio in 2004), for a total of 136,392
cell parameters (about 1 GB of memory) \emph{per simulation}.  It is
thus impossible to store every simulation, or even a substantially
thinned number of simulations, without several terabytes of memory
(supposing that R could handle that much).  We propose to deal with
this memory management issue for higher-dimension data in several ways:  
\begin{itemize}
\item For each iteration (or every iteration saved), the user may
specify a quantity of interest to be calculated and stored (rather
than the parameter draws themselves).   
\item Rather than storing draws in the workspace, the Bayesian methods
will have the option to {\tt sink} draws to a file.  Since these
multi-dimensional data need to be formatted in two dimensions for disk
storage, we will provide functions to reconstruct the higher
dimensions upon reading the sunk file.  
\end{itemize}
In addition, this package will provide wrapper functions to operate on
the margins of higher-dimension arrays, providing useful summary and
print functions.  
\clearpage
\bibliographystyle{apsr}
\bibliography{ei}

\end{document}
